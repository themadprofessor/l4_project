\documentclass[11pt]{article}
\usepackage{times}
\usepackage{fullpage}

\title{The IETF Transport Service API}
\author{Stuart Reilly - 2258082}
\date{\today}

\begin{document}
\maketitle

\section{Status report}

\subsection{Proposal}\label{proposal}

\subsubsection{Motivation}\label{motivation}

Modern network programming requires large amounts of boilerplate code in order
to make the BSD Sockets API usable with modern programming techniques, such
as asynchronous programming and typed messages.
Furthermore, BSD Sockets requires the application to determine the protocol
stack it uses.
This requires each application to have to be updated in order to change
protocol stack, inhibiting uptake of alternative protocols.
The aim of this project is to implement a new high-level network programming
API, which is asynchronous and provides typed message communication.
Such an API would remove large amounts of boilerplate and raise the level in 
the network stack most applications would work with, allowing for alternative
protocols to be used without updating the updating.

\subsubsection{Aims}\label{aims}

The API must be able to provide enough functionality for a simple single-client
and single-server programs.
Where applicable, the API must be asynchronous, through the use of futures.
The Connection object must be able to send and receive typed messages.
These typed messages can be of user-defined types, not just types the API
provides.
There must be a mechanism to frame messages of a known size, exposed by the
API.

\subsection{Progress}\label{progress}

\begin{itemize}
	\item Define the requirements for the API, both functional and
		non-functional.
	\item Begin the background section of the dissertation.
	\item Design high-level traits for the Preconnection, Connection and
		framer objects.
	\item Create tokio-based implementations for the Preconnection and
		Connection traits.
	\item Implement simple connection racing for DNS results.
	\item Create a simple HTTP framer implementation.
	\item Define SelectionProperties and implement Reliability property 
		handling.
	\item Create a small example web client using the tokio implantations
		and the HTTP framer.
\end{itemize}

\subsection{Problems and risks}\label{problems-and-risks}

\subsubsection{Problems}\label{problems}

\begin{itemize}
	\item Rust's async/await syntax not being fully stabilised at the 
		beginning of the project
	\item Libraries changing their API to fit with the new syntax.
	\item Initially implementing with generic types rather than trait 
		objects.
\end{itemize}

\subsubsection{Risks}\label{risks}

\begin{itemize}
	\item Defining how framers handle incomplete messages. i.e.\ waiting for
		more packets.
		
		\emph{Mitigation:} Research existing techniques.
	\item Unsure how best to analyse and evaluate the API and its 
		tokio implementation.

		\emph{Mitigation:} Both discuss with my advisor and research
		API evaluation techniques.
\end{itemize}

\subsection{Plan}\label{plan}

\begin{itemize}
	\item Week 1-2 - Finalise implementation selection mechanism
	\item Week 3-4 - Add unit-tests and more integration tests
	\item Week 5-6 - Finalise code documentation
	\item Week 7-8 - Analyse and evaluate API
	\item Week 9-10 - Submit first draft to advisor
\end{itemize}

\end{document}

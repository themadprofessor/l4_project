In this section, we discuss the high level design of the API, without implementation specific details.
First, a short discussion on the abstract nature of the API .
Followed by the overall flow of the API will be discussed and how each of the main objects connect together.
Then, each of the main objects in the API will be discussed in the order of the rough flow of the API, discussing
the design decisions made for each object.

The API will be expressed primarily through abstract types with multiple possible concrete implementations.
Users of the API will operate almost exclusively with the traits (abstract types) in order to allow the underlying
implementation and protocol stack to change with minimal changes to the user's code.
If the API was defined with concrete types through the use of structs, the underlying implementation which handles
the native network would not be able to change without major changes to the user's code.
Due to Rust's heavy use of monomorphisation and lack of runtime generics, for the API to be able to be dynamically
linked into user applications and distributed separately, the traits will be used through trait objects.
Trait objects are Rust's method of dynamic dispatch and pseudo runtime generics.
While using trait objects would add a performance overhead due to pointer indirection and dynamic dispatch, the benefit
of dynamic linking and alternative distribution out ways this.

\section{Flow}\label{sec:flow}
As shown in Figure~\ref{fig:flow}, there are both asynchronous and synchronous sections of opening a \emph{Connection}.
The asynchronous sections are when the program would have to carry out some I/O, for example binding a \emph{Listener}
to a port, or establishing a \emph{Connection} to a remote device.
The remaining sections are either a trivial object creation or solving properties, as such are synchronous.
Solving properties can contain a concurrent implementation if the implementation deems this necessary.

\begin{figure}[h]
\centering
\includegraphics[width=\textwidth]{diagrams/flow}
\caption{A diagram to show how the three main objects flow from one to another.
Oval nodes are objects, rectangular nodes are states in the transition between objects, dashed arrows are
asynchronous state changes, and solid arrows are synchronous state changes.}
\label{fig:flow}
\end{figure}

\section{Abstract API}\label{sec:abstract-api}
The \emph{Preconnection} type will be defined as a concrete struct, rather than a trait, unlike the \emph{Listener} and
\emph{Connection} types.
Since a \emph{Preconnection} is a configuration type which is independent of the underlying implementation, using a
trait would be redundant as each underlying implementation would implement the trait identically.
Furthermore, due to issues with the Rust compiler, it is difficult to change generic parameters with trait methods.
Listing~\ref{lst:preconnection} shows the definition for the \emph{Preconnection} struct and the generic parameters it
has.
The generic parameters L and R are used to represent the possible local and remote endpoints respectively.
Endpoint state is a marker (empty) trait which is implemented for the empty struct NoEndpoint and any type which
implements the Endpoint trait (See Listing~\ref{lst:endpoint}).
When a \emph{Preconnection} is created, L and R are both NoEndpoint.
Once the \emph{Preconnection} is given a local or remote endpoint, L or R are changed to the type of the given endpoint,
depending on if the endpoint is local or remote.
The initiate and listen methods require L and R respectively to be a type other than NoEndpoint.
This allows the Rust compiler to enforce these constraints, rather than run time checks.
The underlying implementation which provides the \emph{Connection} and \emph{Listener} implementations will be specified
by the I generic parameter, which implements the \emph{Implmentation} trait.
The \emph{Implementation} trait is a primarily an internal trait as it specifies the underlying implementation, but is
public to allow the end user to select from multiple implementations, and create custom implementations.
One method to utilise the \emph{Implementation} is to have the methods take either \texttt{\&mut self} or
\texttt{\&self}, and pass the implementation as a parameter when constructing a \emph{Preconnection}.
This would require the end user to always specify an implementation when constructing a \emph{Preconnection}, even when
using the default implementation.
See Listing~\ref{lst:preconnBad} for an example of this method.
An alternative method is to have the \emph{Implementation} methods not take \texttt{self} in any form, and using it as a
marker.
This would allow the \emph{Preconnection} can specify a default implementation using Rust's default generic parameter.
See Listing~\ref{lst:preconnGood} for two examples of this method, the first \emph{Preconnection} is an example of using
the default implementation, and the second \emph{Preconnection} shows how a custom implementation would be specified.

\begin{lstlisting}[language=Rust, float=h, label=lst:preconnection, caption={The Preconnection struct, showing the four
    generic parameters.}]
pub struct Preconnection<F, L, R, SD, SR, I=DefaultImpl>
where
    F: Framer<SD, SR,
    L: EndpointState,
    R: EndpointState,
    I: Impl
{
    local: L,
    remote: R,
    framer: F,
    trans: TransportProperties,
}
\end{lstlisting}

\begin{lstlisting}[language=Rust, float=h, label=lst:impl, caption={The Implementation trait.}]
pub trait Implementation {
    async fn connection<F, L, R>(
        framer: F,
        local: Option<L>,
        remote: R,
        props: &TransportProperties,
    ) -> Result<Box<dyn Connection<F>>, Error>
    where
        F: Framer + Clone,
        L: Endpoint,
        R: Endpoint;

    async fn listener<F, L, R>(
        framer: F,
        local: L,
        remote: Option<R>,
        props: &TransportProperties,
    ) -> Result<Box<dyn Listener<F, Item = Result<Box<dyn Connection<F>>, Error>>>, Error>
    where
        F: Framer,
        L: Endpoint,
        R: Endpoint;
}
\end{lstlisting}

\begin{lstlisting}[language=Rust, float=h, label=lst:preconnBad, caption={An example of how to construct a
Preconnection if the Implementation trait is to be passed when the Preconnection is constructed}]
fn example() {
    let preconnection = Preconnection::new(Tokio, TransportProperties::default(), ...);
    ...
}
\end{lstlisting}

\begin{lstlisting}[language=Rust, float=h, label=lst:preconnGood, caption={An example of how to construct a
    Preconnection if the Implementation trait is used as a marker.}]
fn example() {
    let preconnection = Preconnection::new(TransportProperties::default(), ...);
    let custom_impl_preconn = Preconnection::<_,_,_,Tokio>::new(TransportProperties::default(), ...);
    ...
}
\end{lstlisting}

The \emph{Connection} will be defined as a trait, and will be constructed with the \emph{Preconnection} object's
initiate method.
As shown in Listing~\ref{lst:connection}, the close and abort methods take \texttt{Box<Self>} rather than \texttt{Self}.
Since \emph{Connection} is to be a trait object, it must be object safe.
An object safe trait, must not require implementations to be sized, not have any methods which consume \texttt{Self}, no
associated constants and all supertraits must be object safe.
As such, for the close and abort methods cannot take \texttt{Self}, instead take \texttt{Box<Self>} since it is sized
and does not require \texttt{Self} to be sized.
These methods consume \texttt{Box<Self>} because the Rust compiler can ensure no other variable has ownership of the
\emph{Connection} when it is closed or aborted, preventing the \emph{Connection} to be used after its closed or aborted.

\begin{lstlisting}[language=Rust, float=h, label=lst:connection, caption={The Connection trait.}]
pub trait Connection<F, S, R>
where
    F: Framer<S, R>
{
    async fn send(&mut self, data: S) -> Result<(), Self::Error>
    where
        S: Encode;

    async fn receive(&mut self) -> Result<R, Self::Error>
    where
        R: Decode;

    async fn close(self: Box<Self>) -> Result<(), Self::Error>;

    fn abort(self: Box<Self>);
}
\end{lstlisting}

The \emph{Listener} trait represents a type which is listening for \emph{Connection}s asynchronously.
One approach to design this trait is to define the trait with a method which returns a \texttt{Future} that resolves
into a Connection when one is received.
An issue with this design is it does not represent a \emph{Listener} as what it is.
Conceptually a \emph{Listener} is an asynchronous stream of \emph{Connection}s, and this design represents a
\emph{Listener} as an object which is idle until the user requests a \emph{Connection}.
At this point it then begins to listen for incoming connections, which could lead to lost connections.
A better design is to have all implementations of the \emph{Listener} trait also implement the \texttt{Stream} trait
from the futures crate.
This allows a \emph{Listener} to make use of the existing \texttt{Stream} ecosystem, such as the \texttt{StreamExt}
trait, which provides an \texttt{Iterator}-style interface to a \texttt{Stream}.
As shown in Listing~\ref{lst:listener}, the implementation for the \texttt{Stream} trait is required to produce an Item
which either a \emph{Connection} or an \emph{Error}.

\begin{lstlisting}[language=Rust, float=h, label=lst:listener, caption={The Listener trait, showing the Stream
    implementation requirement for all implementers.}]
pub trait Listener<F, S, R>: Stream<Item = Result<Box<dyn Connection<F, S, R>>, Error>>
where
    F: Framer,
{
    fn connection_limit(&mut self, limit: usize);
}

\end{lstlisting}

The \emph{Connection} and \emph{Listener} traits and the \emph{Preconnection} struct will be generic over the framer
used to frame the data which the connections will be operating on.
This will ensure the \emph{Connections} can only handle a specific strongly-typed data trait.
All methods on the \emph{Connection} trait which either receive data or send data will utilise this generic parameter in
their return value and parameters respectively.
The \emph{Preconnection}'s listen and initiate methods will pass this generic parameter to the new \emph{Listener}
object or \emph{Connection} object respectively.
An alternative approach is to have the traits generic over the data which the \emph{Connections} will be operating on.
The issue with this approach is the compiler cannot ensure a \emph{Connection} has certain properties which methods
could rely on.
For example, a method could require a \emph{Connection} which is framed with TLS .
If the \emph{Connection} is generic over the \emph{Framer}, the compiler can ensure only \emph{Connections} which are
framed with TLS are used, ensuring the data being transmitted is encrypted.
Furthermore, a collection of traits a \emph{Framer} could implement could be created to ensure only \emph{Connections}
with \emph{Framers} which have specific properties can be used with certain methods.

\subsection{Framer, Encode and Decode}\label{subsec:framer-encode-and-decode}

A \emph{Framer} trait defines a type which can frame and deframe a given piece of data, as well as optionally store
metadata for use in framing and deframing.
The \emph{Framer} must be generic over the data types it will frame and deframe.
By having a \emph{Framer} be generic over both the data it can frame and deframe, different types can be used for input
and output, removing the chance of the end user using the incorrect data.
Initially, the Input and Output types were defined as associated types.
This allowed the implementation specify the accepted types, rather than the end user, which restricts the possibilities
for framers to be generic over arbitrary types for the end user.
For example, a HTTP framer would only be able to send an implementation specific type.
Listing~\ref{lst:framer} shows the Input and Output types as generic types.
The MetaKey and MetaValue types are associated types because they should not be specified by the end user.
All implementations of \emph{Framer} must implement \texttt{Send} and \texttt{Sync}, as well the type definition having
a static lifetime.
\texttt{Send} and \texttt{Sync} are required to allow the \emph{Framer} to be passed between threads and accessed from
different threads.
While a \texttt{Future} does not specify it will be executed on a different thread than it was created, but it allows
it so a \emph{Framer} must be able to be passed to another thread.
Since \emph{Framer}'s methods use \texttt{&mut self}, a \emph{Framer} maybe accessed by a thread other than the thread
which created, so a \emph{Framer} must be able to be accessed from different threads.
Rust allows types to be defined within a function, which restricts the lifetime of these type definitions to the
lifetime of the function.
\texttt{Future}s require all types which are used within the \texttt{Future} must live longer than the \texttt{Future}.
Since a \texttt{Future} can take an indeterminate length of time to complete, all \emph{Framer}s must have a static
lifetime, which means it lives for the lifetime of the program.
An example is a HTTP framer, which can frame a HttpRequest object and deframe a HttpResponse object.
Using different types for framing and deframing means the end user cannot accidentally send the data they have received
back to the sender.
A HttpRequest object and HttpResponse object would contain a type which implements the Encode and Decode trait
respectively.

\begin{lstlisting}[language=Rust, float=h, label=lst:framer, caption={The Framer trait, showing the Send and 'static
    requirements for all implementers, and associated types.}]
pub trait Framer<S, R>: Send + Sync + 'static {
    type MetaKey;
    type MetaValue;
    type Error: StdError + Send;

    fn frame(&mut self, item: S, dst: &mut BytesMut) -> Result<(), Self::Error>
    where
        S: Encode;

    fn deframe(&mut self, src: &[u8]) -> Result<(R, &[u8]), (&[u8], Error<Self::Error>)>
    where
        R: Decode;

    fn add_metadata(&mut self, key: Self::MetaKey, value: Self::MetaValue);
}
\end{lstlisting}

Data which can be framed or deframed must be able to be encoded or decoded from raw bytes if it is to be a
\emph{Framer}'s input and output data trait respectively.
As such they must implement \emph{Encode} trait and/or a \emph{Decode} trait, which define a type which can be encoded
into bytes and decoded from raw bytes respectively.
As shown in Listing~\ref{lst:encode} the \emph{Encode} trait should also provide a hint to the encoded size of the type,
to allow for memory to be preallocated if required.
A default implementation of this method is provided since this is used solely for optimisation.
Both \emph{Encode} and \emph{Decode} will have an associated error type, which must implement both \texttt{Send} and
\texttt{Error}.
This error type will be use for the \texttt{Err} variant of the returned results from the decode or encode methods.
Requiring the type to implement the \texttt{Error} trait ensures the type is a valid error, and the \texttt{Send} trait
ensures the type can be sent between threads, which may occur with \texttt{Future}s.
Allowing an implementation of the \emph{Encode} or \emph{Decode} traits to return their own error type makes
implementing the trait simpler as the implementation does not have to convert their error types into a TAPS error.
There exists the \emph{Serialize} and \emph{Deserialize} traits from the Serde trait, which appear to be similar to
\emph{Encode} and \emph{Decode}.
The aim of Serde's traits it to map between an Rust's data model and Serde's data model.
Serde's data model is designed to be used in conjunction with existing data formats, rather than an arbitrary collection
of bytes.
Therefore, the \emph{Encode} and \emph{Decode} traits are needed to map between typed, structured data and raw bytes.
Serde's trait would be used in \emph{Framers} to convert structured data to and from a particular data format, which is
then framed for sending and deframed when read.

\begin{lstlisting}[language=Rust, float=h, label=lst:encode, caption={The Encode trait, showing the size\_hint method.}]
pub trait Encode {
    type Error: Send + Error;

    fn encode(&self, data: &mut BytesMut) -> Result<(), Self::Error>;

    fn size_hint(&self) -> (usize, Option<usize>) {
        (0, None)
    }
}
\end{lstlisting}

\begin{lstlisting}[language=Rust, float=h, label=lst:decode, caption={The Decode trait.}]
pub trait Decode {
    type Error: Send + Error;

    fn decode(data: &[u8]) -> Result<(&[u8], Self), (&[u8], Error<Self::Error>)>;
}
\end{lstlisting}

In order to handle incomplete data when receiving data from a \emph{Connection}, the error type for both the
\emph{Framer}'s \texttt{deframe} and \emph{Decode}'s \texttt{decode} methods must indicate this.
As shown in Listings~\ref{lst:decode} and~\ref{lst:framer}, the returned error types are wrapped in \texttt{Error},
which is defined as shown in Listing~\ref{lst:codecError}.
This is based of the method used by~\cite{geal_nomerr_nodate} for the nom crate.
If the \emph{Framer} or \emph{Decode} runs out of data while deframing or decoding respectively, it will return a
\texttt{Error::Incomplete}, to signify to the \emph{Connection} to get more data.
Both the \texttt{decode} and \texttt{deframe} methods \texttt{Ok} and \texttt{Err} also contain the remaining data which
has not be used from the input.
The returned bytes will be then passed up to either the \emph{Framer} or \emph{Connection} from the \texttt{decode} or
\texttt{deframe} methods respectively, and used for the next call of the methods where applicable.

\begin{lstlisting}[language=Rust, float=h, label=lst:codecError, caption={The Error type for deframe and decode methods,
    showing the Incomplete varient.}]
pub enum Error<E> {
    Err(E),
    Incomplete
}
\end{lstlisting}

\section{Application Layer Protocols}\label{sec:application-layer-protocols}
Application layer protocols have requirements for the protocol stack they operate on, such as HTTP requiring a reliable
protocol stack.
A \emph{Protocol} type must specify the protocol stack requirements required for the protocol.
This could be represented as a concrete \emph{Protocol} enum, with the variants being the application layer protocols
supported by the API .
The main issue with this approach is only the protocols implemented by the API can be used with the API .
One of the issues discussed with the existing sockets API was how it restricted the uptake of new protocols, which this
approach would replicate, therefore it will not be used.
Another approach is to define an abstract \emph{Protocol} type as a trait, with a method to create a default
\emph{Preconnection} for the protocol.
This would allow third-party crates to add application layer protocols without the API to be updated.

\begin{lstlisting}[language=Rust, float=h, label=lst:protocol, caption={The Protocol trait, add more info here}]
pub trait Protocol {
    fn preconnection() -> Preconnection<F>
    where
        F: Framer;
}
\end{lstlisting}


\section{Connection Racing}\label{sec:connection-racing}
Since the API is protocol-agnostic, there must be a mechanism of determining the protocol stack to use internally.
Multiple protocol stacks can match the same set of transport properties.
Furthermore, a DNS lookup can result in multiple IPv4 and IPv6 addresses.
The combination of all matching protocol stacks and addresses will be raced to find the optimal stack.
Protocol stacks will be ordered by how preferred they are according to the transport properties.
For example, if the transport properties specifies reliability is preferred, reliable protocol stacks are ordered before
unreliable protocol stacks.
IP addresses received from a DNS lookup will be ordered according to the Happy Eyeballs version 2 algorithm.
This specifies IPv4 and IPv6 addresses should be resolved asynchronously, prioritising IPv6 responses.
If an IPv6 address resolves first, attempt to connect to the address, falling back to an IPv4 address if the connection
failed.
If an IPv4 address resolves first, delay the connection attempt until either an IPv6 address resolves and connects
successfully, or period of time (Resolution Delay Period) has passed.
Furthermore, prioritise addresses resolved from a DNS server with an IPv6 address.
Once an address successfully connects, close connections still connecting, and do not attempt anymore connections.
Utilising Happy Eyeballs version 2 the implementation to dynamically select the optimal address to connect to, while
minimising the number of connections which need to be created to find the optimum.

%==================================================================================================================================
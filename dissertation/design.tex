In this section, we discuss the high level design of the API, without implementation specific details.
First, a short discussion on the abstract nature of the API .
Followed by the overall flow of the API will be discussed and how each of the main objects connect together.
Then, each of the main objects in the API will be discussed in the order of the rough flow of the API, discussing
the design decisions made for each object.

\section{Flow}\label{sec:flow}
\begin{figure}[h]
    \centering
    \includegraphics[width=\textwidth]{diagrams/flow}
    \caption{A diagram to show how the three main objects flow from one to another.
    Oval nodes are objects, rectangular nodes are states in the transition between objects, dashed arrows are
    asynchronous state changes, and solid arrows are synchronous state changes.}
    \label{fig:flow}
\end{figure}

\begin{figure}[h]
    \centering
    \includegraphics[width=\textwidth]{diagrams/lidecycle.png}
    \caption{A diagram to show a high-level view of the actions that occur during the lifecycle of a Connection object.
    Method calls in italics are internal calls.}
    \label{fig:lifecycle}
\end{figure}

As shown in Figure~\ref{fig:flow}, there are both asynchronous and synchronous sections of opening a \connection{}.
The asynchronous sections are when the program would have to carry out some I/O, for example binding a \listener{}
to a port, or establishing a \connection{} to a remote device.
The remaining sections are either a trivial object creation or solving properties, as such are synchronous.
Solving properties can contain a concurrent implementation if the implementation deems this necessary.
Figure~\ref{fig:lifecycle} shows the lifecycle of a \connection{} object.
All \connection{} objects are created either directly via \texttt{initiate()} or indirectly via a \listener{}.
A \connection{} then communicates with the network through a \framer{}, until \connection{} is no longer
needed and \texttt{close()} or \texttt{abort()} are called.
Rust's \texttt{Drop} allows the user of the API to not call either \texttt{close()} or \texttt{abort()} explicitly, as
it allows arbitrary code to be executed when a variable goes out of scope.
Since utilising \texttt{Drop} does not allow for errors to be propagated back to the user, it will likely be similar to
a \texttt{close()} call where all errors are silently discarded.
Section~\ref{subsec:connection} describes the difference between \texttt{close()} and \texttt{abort()}.
See Section~\ref{subsec:framer} for details on why a \framer{} is required for a \connection{}.

\section{Abstract API}\label{sec:abstract-api}
The API will be expressed primarily through abstract types with multiple possible concrete implementations.
Users of the API will operate almost exclusively with the traits (abstract types) in order to allow the underlying
implementation and protocol stack to change with minimal changes to the user's code.
If the API was defined with concrete types through the use of structs, the underlying implementation which handles
the native network would not be able to change without major changes to the user's code.
Due to Rust's heavy use of monomorphisation and lack of runtime generics, for the API to be able to be dynamically
linked into user applications and distributed separately, the traits will be used through trait objects.
Trait objects are Rust's method of dynamic dispatch and pseudo runtime generics.
All trait objects must be wrapped in \texttt{Box} and marked as \texttt{dyn}, for example
\texttt{Box<dyn Connection<F>>}, which leads to cluttered code.
This can be mediated through type alias, so for example if \texttt{type ConnectionObj<F> = Box<dyn Connection<F>>;} was
defined, \texttt{ConnectionObj<F>} can be used instead.
While using trait objects would add a performance overhead due to pointer indirection and dynamic dispatch, the benefit
of dynamic linking and alternative distribution out ways this.

\subsection{Preconnection}\label{subsec:preconnection}
First the user will create a \preconnection{}, which will be used to define the properties a \connection or \listener
must have.
A \preconnection{} object is a configuration object.
It contains the endpoints, both local and remote, a \connection{} or \listener{} should connect or bind too
respectively, as well as the properties the resultant \connection{} or \listener{} should have.
\cite[§~5.2]{trammell_abstractapplicationlayer_2020} defines a set of properties a \preconnection{} can
configure.

\begin{lstlisting}[float=h, label=lst:preconnection, caption={The Preconnection struct, showing the four
generic parameters.}]
pub struct Preconnection<F, L, R, I=DefaultImpl>
where
    F: Framer,
    L: EndpointState,
    R: EndpointState,
    I: Implementation
{
    local: L,
    remote: R,
    framer: F,
    trans: TransportProperties,
}
\end{lstlisting}

The \preconnection{} type will be defined as a concrete struct, rather than a trait, unlike the \listener{} and
\connection{} types.
Since a \preconnection{} is a configuration type which is independent of the underlying implementation, using a
trait would be redundant as each underlying implementation would implement the trait identically.
Furthermore, due to issues with the Rust compiler, it is difficult to change generic parameters with trait methods,
which is used to ensure endpoints are specified at compile time.
Listing~\ref{lst:preconnection} shows the definition for the \preconnection{} struct and the generic parameters it
has.
The generic parameters L and R are used to represent the possible local and remote endpoints respectively.
See Section~\ref{sec:preconnection-endpoint-safety} for details on these parameters.

\subsection{Implementation}\label{subsec:implementation}
While the user should not focus on the underlying implementation of the traits, it needs to be defined for the
\preconnection{} to know how to instantiate them.
The \emph{Implementation} trait is a primarily an internal trait as it specifies the underlying implementation, but is
public to allow the end user to select from multiple implementations, and create custom implementations.
The specific implementation will be given \preconnection{}'s \texttt{I} generic parameter, which implements the
\emph{Implmentation} trait.

\begin{lstlisting}[float=h, label=lst:impl, caption={The Implementation trait.}]
pub trait Implementation {
    async fn connection<F, L, R>(
        framer: F,
        local: Option<L>,
        remote: R,
        props: &TransportProperties,
    ) -> Result<Box<dyn Connection<F>>, Error>
    where
        F: Framer + Clone,
        L: Endpoint,
        R: Endpoint;

    async fn listener<F, L, R>(
        framer: F,
        local: L,
        remote: Option<R>,
        props: &TransportProperties,
    ) -> Result<Box<dyn Listener<F, Item = Result<Box<dyn Connection<F>>, Error>>>, Error>
    where
        F: Framer,
        L: Endpoint,
        R: Endpoint;
}
\end{lstlisting}

One method to utilise the \emph{Implementation} is to have the methods take either \texttt{\&mut self} or
\texttt{\&self}, and pass the implementation as a parameter when constructing a \preconnection{}.
This would require the end user to always specify an implementation when constructing a \preconnection{}, even when
using the default implementation.
See Listing~\ref{lst:preconnBad} for an example of this method.

\begin{lstlisting}[float=h, label=lst:preconnBad, caption={An example of how to construct a
Preconnection if the Implementation trait is to be passed when the Preconnection is constructed}]
fn example() {
    let preconnection = Preconnection::new(Tokio, TransportProperties::default(), ...);
    ...
}
\end{lstlisting}

An alternative method is to have the \emph{Implementation} methods not take \texttt{self} in any form, and using it as a
marker.
This would allow the \preconnection{} can specify a default implementation using Rust's default generic parameter
mechanism.
See Listing~\ref{lst:preconnGood} for two examples of this method, the first \preconnection{} is an example of using
the default implementation, and the second \preconnection{} shows how a custom implementation would be specified.
Since the underlying implementation is not intended to be a focus for the user, this method will be used.

\begin{lstlisting}[float=h, label=lst:preconnGood, caption={An example of how to construct a
Preconnection if the Implementation trait is used as a marker.}]
fn example() {
    let preconnection = Preconnection::new(TransportProperties::default(), ...);
    let custom_impl_preconn = Preconnection::<_,_,_,Tokio>::new(TransportProperties::default(), ...);
    ...
}
\end{lstlisting}

\subsection{Endpoint}\label{subsec:endpoint}
A \connection{} and \listener{} need to know what to connect to or listen to respectively, and this is specfied in the
creating \preconnection{} by giving it either local or remote \Endpoint{}s.
An \Endpoint{} is a type which represents a possible either a local or remote endpoint.
This could be a string containing a domain name, or an IPv6 address and port number for example.
It does not represent an endpoint, rather a possible endpoint which needs to be resolved into a \texttt{SocketAddr}.

\begin{lstlisting}[float=h, label=lst:endpoint, caption={The Endpoint Trait with the resolve method returning an
        Iterator.}]
pub trait Endpoint: Send + 'static {
    type Error: Send + Error;
    type Iter: Iterator<Item = SocketAddr> + Send;

    async fn resolve(self) -> Result<Self::Iter, Self::Error>;
}
\end{lstlisting}

As shown in Listing~\ref{lst:endpoint}, \Endpoint{} requires a \texttt{resolve} method.
This method is where a possible endpoint is resolved into a \texttt{SocketAddr}.
There are no implementations of the trait provided as Rust does not allow a trait to be implemented twice for the same
type.
A trivial implementation could be provided for the \texttt{SocketAddr}, but implementations with traits similar to
Rust's \texttt{ToSocketAddrs} trait, which have implementations for \texttt{SocketAddr}, cannot be used as a blanket
implementation for \Endpoint{}.
All implementations of \Endpoint{} must also be \texttt{Send} since \texttt{resolve} is async, which means it will
make use of a \texttt{Future}, which could be ran on a different thread.
The associated types \texttt{Error} and \texttt{Iter} require the type to be \texttt{Send} for the same reason.
Furthermore, since there is no guarantee as to how long the \texttt{Future} will take to complete, the type which
implements \Endpoint{} must exist for the length of the program, hence must have a \texttt{'static} lifetime.
Upon a successful resolution, the resolved \texttt{SocketAddr}s are returned as an \texttt{Iterator}.
Initially \texttt{resolve} returned a \texttt{Vec<SocketAddr>}, primarily because it allowed for simpler type defintions
and didn't require an associated type.
Using a \texttt{Vec} requires explicit allocation, which would lead to an unnecessary performance hit when an
implementation operates on an \texttt{Iterator} internally or is a trivial implementation such as for
\texttt{SocketAddr}.
As such, \texttt{resolve} was modified to return an \texttt{Iterator}, which was defined as the associated type
\texttt{Iter}.
This can lead to a long type definition for \texttt{Iter}, but this can be mediated with the unstable feature
\texttt{type\_alias\_impl\_trait}\footnote{\url{https://github.com/rust-lang/rust/issues/63063}}.

\subsection{Connection}\label{subsec:connection}
Once the \preconnection{} is fully configured, it's \texttt{initate()} method can be called to initate connection
establish ment to the \preconnection{}s remote \Endpoint{}.
A \connection{} object represents a messaged-orientated connection between a local and remote endpoint.
It wraps an underlying protocol stack the user of the API specifies with the \emph{Implementation}
parameter of the \preconnection{} which crated the \connection{}.
The user shouldn't need to focus on the underlying protocol stack as the \texttt{initiate()} method and
\listener{} will ensure the required properties for the \connection{} are upheld.
A \connection{} cannot be created without a remote endpoint.
Without a remote endpoint, a \connection{} has nothing to connect too.
The remote endpoint can either be given to the creating \preconnection{} for use in the \texttt{initate()} method,
or from the incoming \emph{connection} a \listener{} receives.
If the creating \preconnection{} is given a local endpoint, the \texttt{initaite()} method will ensure the resulting
\connection{} is between the required remote endpoint and given local endpoint.
By default, the \texttt{initate()} method will attempt to establish a \emph{connection} between the required remote
endpoint and any applicable local endpoint.

\begin{lstlisting}[float=h, label=lst:connection, caption={The Connection trait.}]
pub trait Connection<F, S, R>
where
F: Framer<S, R>
{
    async fn send(&mut self, data: S) -> Result<(), Self::Error>
    where
        S: Encode;

    async fn receive(&mut self) -> Result<R, Self::Error>
    where
        R: Decode;

    async fn close(self: Box<Self>) -> Result<(), Self::Error>;

    fn abort(self: Box<Self>);
}
\end{lstlisting}

The \connection{} will be defined as a trait, and will be constructed with the \preconnection{} object's
initiate method.
Each implementation of the \emph{Implementation} trait will have its own underlying representation of a connection
between a local and remote endpoint.
Defining a \connection{} as trait allows the user's code to use any underlying representation of a connection
without having to specifically handle each possible representation.
As shown in Listing~\ref{lst:connection}, the close and abort methods take \texttt{Box<Self>} rather than \texttt{Self}.
Since \connection{} is to be a trait object, it must be object safe.
An object safe trait, must not require implementations to be sized, not have any methods which consume \texttt{Self}, no
associated constants and all supertraits must be object safe.
As such, for the close and abort methods cannot take \texttt{Self}, instead take \texttt{Box<Self>} since it is sized
and does not require \texttt{Self} to be sized.
These methods consume \texttt{Box<Self>} because the Rust compiler can ensure no other variable has ownership of the
\connection{} when it is closed or aborted, preventing the \connection{} to be used after its closed or aborted.

Both \texttt{close} and \texttt{abort} methods are provided because a \texttt{close} will gracefully close the
underlying connection.
For connection-oriented transport protocols, \texttt{close} should inform the remote endpoint the connection
is closed, allowing for resource cleanup.
Connection-less transport protocols may inform the remote endpoint if applicable.
Other the other hand, \texttt{abort} should cleanup the local resources, but makes no guarantee about if the remote
endpoint is informed of the connection closing.
If a \connection{} has a \texttt{Drop} implementation, it should have similar semantics to \texttt{close}, but
discard errors produced.

\subsection{Listener}\label{subsec:listener}
Once the \preconnection{} is fully configured, it's \texttt{listen()} method can be called to bind a \listener{} to the
\preconnection{}'s local \Endpoint{}.
A \listener{} represents a stream of incoming \connection{}s for a local endpoint.
It is bound to the local endpoint in the creating \preconnection{}, and listens for incoming \connection{}s,
represented as an asynchronous stream of \emph{Connections}.
A \listener{} cannot be created without a local endpoint.
Without a local endpoint, the \listener{} cannot be bound and cannot listen ofr incoming \connection{}s.
The local endpoint must be given to the creating \preconnection{} for use in the \texttt{listen()} method.
A remote endpoint is optional, and if it is given to the creating \preconnection{}, the resulting \listener{} will only
accept \connection{}s from the given remote endpoint.
A \listener{} created without a remote endpoint will accept any incoming \connection{}s.

\begin{lstlisting}[float=h, label=lst:listener, caption={The Listener trait, showing the Stream
implementation requirement for all implementers.}]
pub trait Listener<F, S, R>: Stream<Item = Result<Box<dyn Connection<F, S, R>>, Error>>
where
    F: Framer + Clone,
{
    fn connection_limit(&mut self, limit: usize);
}

\end{lstlisting}
One approach to design this trait is to define the trait with a method which returns a \texttt{Future} that resolves
into a Connection when one is received.
An issue with this design is it does not represent a \listener{} as what it is.
Conceptually a \listener{} is an asynchronous stream of \connection{}s, and this design represents a
\listener{} as an object which is idle until the user requests a \connection{}.
At this point it then begins to listen for incoming connections, which could lead to lost connections.
A better design is to have all implementations of the \listener{} trait also implement the \texttt{Stream} trait
from the futures crate.
This allows a \listener{} to make use of the existing \texttt{Stream} ecosystem, such as the \texttt{StreamExt}
trait, which provides an \texttt{Iterator}-style interface to a \texttt{Stream}.
As shown in Listing~\ref{lst:listener}, the implementation for the \texttt{Stream} trait is required to produce an Item
which either a \connection{} or an \emph{Error}.
For a \framer{} to be given to a \listener{}, a new version of it needs to be given to each \connection{}
received.
As such, \framer{}'s which can be used with the \listener{} trait must implement \texttt{Clone}.
Since \texttt{Clone}'s \texttt{clone} method allows arbitrary code, \framer{}s which contain information which
shouldn't be duplicated rather reconstructed, the implementation can do that as required.

The \connection{} and \listener{} traits and the \preconnection{} struct will be generic over the framer
used to frame the data which the connections will be operating on.
This will ensure the \connection{}s can only handle specific strongly-typed data.
All methods on the \connection{} trait which either receive data or send data will utilise this generic parameter in
their return value and parameters respectively.
The \preconnection{}'s listen and initiate methods will pass this generic parameter to the new \listener{}
object or \connection{} object respectively.
An alternative approach is to have the traits generic over the data which the \connection{}s will be operating on.
The issue with this approach is the compiler cannot ensure a \connection{} has certain properties which methods
could rely on.
For example, a method could require a \connection{} which is framed with TLS .
If the \connection{} is generic over the \framer{}, the compiler can ensure only \connection{}s which are
framed with TLS are used, ensuring the data being transmitted is encrypted.
Furthermore, a collection of traits a \framer{} could implement could be created to ensure only \connection{}s
with \framer{}s which have specific properties can be used with certain methods.

\subsection{Framer}\label{subsec:framer}
When a \connection{}'s \texttt{send()} method is called, the message it's given is framed by the \connection{}'s
\framer{} before being passed to the underlying transport layer.
Similiarly, during a \connection{}'s \texttt{recv()} method, when data is received, the message is deframed from the
raw bytes by the \connection{}'s \framer{}.
\begin{lstlisting}[float=h, label=lst:framer, caption={The Framer trait, showing the Send and 'static
requirements for all implementers, and associated types.}]
pub trait Framer<S, R>: Send + Sync + 'static {
    type Input: Encode + Send;
    type Output: Decode;
    type MetaKey;
    type MetaValue;
    type Error: StdError + Send;

    fn frame(&mut self, item: Self::Input, dst: &mut BytesMut) -> Result<(), Self::Error>
    where
        S: Encode;

    fn deframe(&mut self, src: &mut BytesMut) -> Result<Self::Output, DeframeError<Self::Error>
    where
        R: Decode;

    fn add_metadata(&mut self, key: Self::MetaKey, value: Self::MetaValue);
}
\end{lstlisting}

A \framer{} trait defines a type which can frame and deframe a given piece of data, as well as optionally store
metadata for use in framing and deframing.
Datagram-orientated transport protocols like UDP have defined boundaries for messages, therefore don't require a method
of defining these boundaries.
Stream-orientated transport protocols like TCP do not have defined boundaries as they present a stream of bytes.
A \framer{} defines the message boundaries, for example prepending the message with the length of the message or
defining an end-of-message sequence.

The \framer{} must be generic over the data types it will frame and deframe.
By having a \framer{} be generic over both the data it can frame and deframe, different types can be used for input
and output, removing the chance of the end user using the incorrect data.
The Input and Output types are defined as associated types.
This allowed the implementation specify the accepted types, rather than the end user, which allows an implementation to
restrict what data it can send and receive.
An alternative approach is to defined Input an Output as generic types on the frame and deframe methods respectively.
This would allow the end user to pass an arbitrary type which implements a trait, which could be specified the by
the implementation.
Such an approach would make it very difficult to work with header information.
For example, a HTTP framer would not be able to return the status code to the end user.
Associated type mean the implementation can limit the types the end user can use to type which handle such header data.
These types can contain their own generic parameters, for example a HTTPResponse object could have a generic type for
the body of the response, which can be restricted to types which implement \decode{}.
Listing~\ref{lst:framer} shows the Input and Output types as associated types.
The MetaKey and MetaValue types are associated types because they should not be specified by the end user.

All implementations of \framer{} must implement \texttt{Send} and \texttt{Sync}, as well the type definition having
a static lifetime.
\texttt{Send} and \texttt{Sync} are required to allow the \framer{} to be passed between threads and accessed from
different threads.
While a \texttt{Future} does not specify it will be executed on a different thread than it was created, but it allows
it so a \framer{} must be able to be passed to another thread.
Since \framer{}'s methods use \texttt{\&mut self}, a \framer{} maybe accessed by a thread other than the thread
which created, so a \framer{} must be able to be accessed from different threads.
Rust allows types to be defined within a function, which restricts the lifetime of these type definitions to the
lifetime of the function.
\texttt{Future}s require all types which are used within the \texttt{Future} must live longer than the \texttt{Future}.
Rust allows type to be defined within functions, so the type definition would live until the function returned.
Since a \texttt{Future} can take an indeterminate length of time to complete, all \framer{} type definitions must
have a static lifetime, which means it lives for the lifetime of the program.

An example is a HTTP framer, which can frame a HttpRequest object and deframe a HttpResponse object.
Using different types for framing and deframing means the end user cannot accidentally send the data they have received
back to the sender.
A HttpRequest object and HttpResponse object would contain a type which implements the \encode{} and \decode{}
trait respectively.

\subsection{Encode and Decode}\label{subsec:encode-and-decode}
During a \connection{}'s \texttt{send()} method, the \connection{}'s \framer{} frames the message \texttt{send()} is
given, and the \framer{} calls the message's \texttt{encode()} method to convert the message into raw bytes to be given
to the underlying transport protocol.
Similarly, the \connection{}'s \framer{} is called during the \connection{}'s \texttt{recv()} method once data is
received, which first deframes the message before calling the message's \decode{} method to attempt to convert the
raw bytes into a message.
The \connection{} sends and recieves typed messages, but the underlying protocol stack operates on raw bytes,
therefore messages must be able to be converted to and from raw bytes.
\encode{} defines a type which can be encoded into bytes for transport.
\decode{} defines a type which can be decoded from bytes, and support incomplete data when decoding.

\begin{lstlisting}[float=h, label=lst:encode, caption={The Encode trait, showing the size\_hint method.}]
pub trait Encode {
    type Error: Send + Error;

    fn encode(&self, data: &mut BytesMut) -> Result<(), Self::Error>;

    fn size_hint(&self) -> (usize, Option<usize>) {
        (0, None)
    }
}
\end{lstlisting}

\begin{lstlisting}[float=h, label=lst:decode, caption={The Decode trait.}]
pub trait Decode {
    type Error: Send + Error;

    fn decode(data: &mut BytesMut) -> Result<Self, Error<Self::Error>>;
}
\end{lstlisting}

As shown in Listing~\ref{lst:encode} the \encode{} trait should also provide a hint to the encoded size of the type,
to allow for memory to be preallocated if required.
A default implementation of this method is provided since this is used solely for optimisation.
Both \encode{} and \decode{} will have an associated error type, which must implement both \texttt{Send} and
\texttt{Error}.
This error type will be use for the \texttt{Err} variant of the returned results from the decode or encode methods.
Requiring the type to implement the \texttt{Error} trait ensures the type is a valid error, and the \texttt{Send} trait
ensures the type can be sent between threads, which may occur with \texttt{Future}s.
Allowing an implementation of the \encode{} or \decode{} traits to return their own error type makes
implementing the trait simpler as the implementation does not have to convert their error types into a TAPS error.
There exists the \emph{Serialize} and \emph{Deserialize} traits from the Serde trait, which appear to be similar to
\encode{} and \decode{}.
The aim of Serde's traits it to map between an Rust's data model and Serde's data model.
Serde's data model is designed to be used in conjunction with existing data formats, rather than an arbitrary collection
of bytes.
Therefore, the \encode{} and \decode{} traits are needed to map between typed, structured data and raw bytes.
Serde's trait would be used in \emph{Framers} to convert structured data to and from a particular data format, which is
then framed for sending and deframed when read.

\begin{lstlisting}[float=h, label=lst:codecError, caption={The Error type for deframe and decode methods,
showing the Incomplete varient.
It is often aliased to DecodeError.}]
pub enum Error<E> {
    Err(E),
    Incomplete
}
\end{lstlisting}

In order to handle incomplete data when receiving data from a \connection{}, the error type for both the
\framer{}'s \texttt{deframe} and \decode{}'s \texttt{decode} methods must indicate this.
As shown in Listings~\ref{lst:decode} and~\ref{lst:framer}, the returned error types are wrapped in \texttt{Error},
which is defined as shown in Listing~\ref{lst:codecError}.
This is based of the method used by~\cite{geal_nomerrrust_} for the nom crate.
If the \framer{} or \decode{} runs out of data while deframing or decoding respectively, it will return a
\texttt{Error::Incomplete}, to signify to the \connection{} to get more data.
Both the \texttt{decode} and \texttt{deframe} methods \texttt{Ok} and \texttt{Err} also contain the remaining data which
has not be used from the input.
Using \texttt{BytesMut} allows the implementations of \decode{} and \framer{} to advance the start of the
current buffer internally, and allows the \connection{} implementation to continue to read more data into the same
buffer.
For example, a HTTP framer would attempt to read until the end of a header line.
If it successfully finds the end of line, it can advance the \texttt{ByteMut} to the end of line to avoid parsing the
line again.

\section{Application Layer Protocols}\label{sec:application-layer-protocols}
\begin{lstlisting}[float=h, label=lst:protocol, caption={The Protocol trait.}]
pub trait Protocol {
    fn preconnection() -> Preconnection<F>
    where
        F: Framer;
}
\end{lstlisting}

Application layer protocols have requirements for the protocol stack they operate on, such as HTTP requiring a reliable
protocol stack.
A \emph{Protocol} type must specify the protocol stack requirements required for the protocol.
This could be represented as a concrete \emph{Protocol} enum, with the variants being the application layer protocols
supported by the API .
The main issue with this approach is only the protocols implemented by the API can be used with the API .
One of the issues discussed with the existing sockets API was how it restricted the uptake of new protocols, which this
approach would replicate, therefore it will not be used.
Another approach is to define an abstract \emph{Protocol} type as a trait, with a method to create a default
\preconnection{} for the protocol.
This would allow third-party crates to add application layer protocols without the API to be updated.

\section{Connection Racing}\label{sec:connection-racing}
Since the API is protocol-agnostic, there must be a mechanism of determining the protocol stack to use internally.
Multiple protocol stacks can match the same set of transport properties.
Furthermore, a DNS lookup can result in multiple IPv4 and IPv6 addresses.
The combination of all matching protocol stacks and addresses will be raced to find the optimal stack.
Protocol stacks will be ordered by how preferred they are according to the transport properties.
For example, if the transport properties specifies reliability is preferred, reliable protocol stacks are ordered before
unreliable protocol stacks.
IP addresses received from a DNS lookup will be ordered according to the Happy Eyeballs version 2 algorithm.
This specifies IPv4 and IPv6 addresses should be resolved asynchronously, prioritising IPv6 responses.
If an IPv6 address resolves first, attempt to connect to the address, falling back to an IPv4 address if the connection
failed.
If an IPv4 address resolves first, delay the connection attempt until either an IPv6 address resolves and connects
successfully, or period of time (Resolution Delay Period) has passed.
Furthermore, prioritise addresses resolved from a DNS server with an IPv6 address.
Once an address successfully connects, close connections still connecting, and do not attempt anymore connections.
Utilising Happy Eyeballs version 2 the implementation to dynamically select the optimal address to connect to, while
minimising the number of connections which need to be created to find the optimum.

%==================================================================================================================================
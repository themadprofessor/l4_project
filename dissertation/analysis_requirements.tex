The project will aim to produce a high-level, typed and asynchronous networking API, based on the IETF's TAPS, in the
Rust programming language, making use of Rust's \asyncawait{} syntax and \texttt{Future}s.
This API is intended to be a replacement for BSD sockets, and as such will provide similar functionality, albeit at a
higher level.
Since the project will be based on TAPS, \preconnection{}, \framer{}, \listener{} and \connection{} in this chapter
follow the definition discussed in Section~\ref{sec:taps}, to be fully defined in Section~\ref{sec:abstract-api}.

In this chapter we define a set of functional and non-functional requirements, splitting each into ''required'' and
''optional''.
These requirements must fulfill the aims outlined in Chapter~\ref{ch:introduction} and are intended to be built upon and
make use of the prior research discussed in Chapter~\ref{ch:background}.

\section{Functional Requirements}\label{sec:functional-requirements}

\subsection{Required}\label{subsec:required}
The \preconnection{} must allow for the user to specify both the local and remote machine the resulting \connection{}
will operate with.
Without this functionality, a \connection{} would not be able to be established as there would be nothing to establish
between.

The API must provide enough functionality to support a single-client and single-server example.
This is required, because this is a minimum requirement to be able to be a basic replacement for BSD sockets.
Furthermore, most network communication follow this style, and the remainder can be recreated using this style.
As such a \preconnection{} must be defined which can initiate a \connection{} object, or begin listening for
\connection{}s.

The API must be asynchronous, utilising Futures as the mechanism of handling asynchronous events.
This is required, because one the main issues with BSD sockets discussed in Chapter~\ref{ch:background} was its
synchronous nature.
Asynchronous network programming is the modern technique, as such must be the primary method of using a new networking
API .
Futures are the de-facto standard method for asynchronous programming in Rust, as such they shall be used.

The API must define a \connection{} object which can send and receive typed data.
As discussed in Chapter~\ref{ch:background}, one of the issues with BSD sockets is they only support sending byte
streams or byte datagrams for TCP and UDP sockets respectively.
As such the API must expose a mechanism for typed data to be passed to a \connection{} object, which will be converted
into the appropriate byte structure by the \connection{} object.

The API must define a mechanism to frame messages of a known size.
This is required, because to transform underlying transport layer data to and from useful message data, it must be
framed.
Without a framing mechanism, a byte-stream based transport layer will not be able to send and receive typed data,
as defined by the previous requirement.

\subsection{Optional}\label{subsec:optional}

The API could provide a mechanism to frame messages of an unknown size.
This would allow byte-stream based transport layer data to be transformed to and from useful message data, even without
knowing the size of the data being transferred.
Since this section of the TAPS specification is not well-defined and is under continuous modification, this will not be
required to be implemented by this project.

The API could provide a \connection{} object which can receive partial data.
Since the underlying transport layer could not provide reliable data transfer, the \connection{} object could provide
partial data, which could be completed with subsequent packets.
This section of the TAPS specification is currently not well-defined and is under continuous modification, this will not
be required to be implemented by this project.

The API could provide a mechanism to enforce \preconnection{} object validity at compile-time through the use of Rust's
type system.
A \preconnection{} requires a remote endpoint or local endpoint to initiate a \connection{} or begin listening for
\connection{}s, respectively.
This can be implemented through a runtime error trivially, but to ensure the API is as safe as possible, this could be
implemented as a compile-time error.
Since implementing this as a compile-time error is non-trivial, and does not impact the usability of the API, this will
not be required to implement this project.

\section{Non-Functional Requirements}\label{sec:non-functional-requirements}

\subsection{Required}\label{subsec:required2}

The API must expose a “self-documenting” API, meaning the names of methods, method arguments, objects and constants must be
meaningful and describe their use.
This ensures the API describes how to use itself through its naming, using explicit documentation to augment this description.
Furthermore, this lowers the learning curve of the API, possibly increasing the rate of adoption.

The API must allow users of the API to provide their own types for sending and receiving over a \connection{}.
Without this mechanism, the API will be very restrictive and not useful.
The Rust ecosystem has support for arbitrary type serialisation and deserialisation through the Serde library, as such this
will likely be used to provide this mechanism.

\subsection{Optional}\label{subsec:optional2}

The API should not have a large impact performance over raw BSD sockets.
A large performance impact will hinder the adoption of the API, and as such should be avoided.
Since this is primarily a proof-of-concept project, this is not required, but should still be taken into account.

%==================================================================================================================================
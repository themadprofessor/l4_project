\section{Rust}\label{sec:rust}
Rust is a modern systems programming language, which aids developers “write faster, more reliable software”, according
to the Rust book (\cite{kalbnik_rustprogramminglanguage_}).
Rust targets systems level programming, similar to C and C++, but provides much greater safety in comparison.

One major issue with writing system software is ownership.
Who owns a resource, such as memory or file handles, and for how long.
In higher level languages, a garbage collector will ensure no resources live longer than they are needed, often through
the use of reference counting.
Garbage collection does elevate the burden of resource management from the developer, but in return, the developer has
much less control over resources, and the resources will live of a non-deterministic length of time with most garbage
collectors.
Languages like C and C++ require the programmer to manage resources, releasing them manually, leading to unreleased
resources if implemented incorrectly.
Rust enforces correct resource management by delegating all resource management to the compiler, through its borrow
checker.
Only a single entity can own a resource in Rust.
All other entities must use references to interact with the resource.
Furthermore, either a single mutable reference or an arbitrary number of immutable references can exist for a resource
at once.
These references cannot live longer than the entity owning the resource, which is enforced by the compiler.
When a reference goes out of scope, it is freed.
When an entity goes out of scope, there must be no remaining references to the entity, and the resources the entity owns
are freed.
Through this method, invalid references cannot exist, and it is impossible to release a resource which has already been
released.
A side effect of only allowing a single mutable reference, or an arbitrary number of immutable references, the compiler
can guarantee a compiled program is data race free.
This is because, only a single entity can modify a resource, while knowing no other entities can read the resource or
modify the resource.

Being a modern language, Rust provides some higher level concepts and construct to systems programming.
While Rust is a statically typed language, it provides local type inference, allowing for cleaning code.
Its type-system is inspired by other languages such as Haskell, as such a programmer can express a large amount of
information through the types they use alone.
Furthermore, Rust has support for higher-order functions, because functions are a first-class type.
Functions, such as map and fold, are provided, not directly for collections of items, rather iterators.
Iterators in Rust provide a functional method for working with collections of items, while providing performance similar
to a comparable loop construct.

Rust also emphasises its ability to provide “zero-cost abstractions”.
This means using high level constructs, such as iterators, do not inure any performance cost over using lower-level
constructs, allowing for code to be written faster and cleaner, without any performance impact.
Another example is Rust's Option type, which is either Some wrapping an object, or None.
This can be the equivalent to a \texttt{NULL}-pointer check in C, due to Rust's null-pointer optimisation, but is much
safer as the type-system forces the programmer to check for the presence of an object.
A Future in Rust is a type which is polled until it is complete.
A key feature of Rust's Futures is they only carry out work when polled, so there is no extra cost to use a Future.
Rather than polling all the Futures manually, a runtime is used to manage the Futures.
The runtime is responsible for distributing the Futures among its worker poll if applicable, and ensuring each Future is
polled.

Recently, Rust introduced support for \texttt{async/await} syntax~\citep{withoutboats_asyncawaitnotation_}.
A function can be marked as async, which changes the return type to be a Future wrapping the original return type.
When an \texttt{async} function reaches an \texttt{await}, the poll being used to progress the Future returns Pending,
informing the runtime this Future is incomplete and another Future should be polled.
The Future will continue to return Pending until the Future being awaited completes, at which point the function will
continue until the next await or the function returns.
The new syntax allows for asynchronous code to be written quickly and cleanly with no runtime cost, compared to manually
handling Futures.

\section{NEAT}\label{sec:neat}
The NEAT API~\citep{khademi_neatplatformprotocolindependent_2017} is a library to provide a platform and protocol
agnostic network access.
It removes large amounts of existing boilerplate networking code, such as DNS resolution, and decouples the application
from the protocols used.
This allows for smaller application code bases, and increased uptake in alternative protocols.
Existing network applications must specify the protocols they wish to use.
NEAT does not allow the application to do this, rather NEAT determines a collection of suitable protocols internally.
The collection is iterated over, attempting to open a connection with each protocol.
If a connection is not established within a timeout, the next protocol is tried, but the existing attempts are not
cancelled, rather the first successful connection attempt is used, irrespective of when it was attempted.
This allows for alternative protocols to be attempted before conventional protocols, with minimal delay if the
alternative protocols fail to connect.

\section{TAPS}\label{sec:taps}
Transport Services~\citep{pauly_architecturetransportservices_2019} is proposed standard from the TAPS working group.

%What did other people do, and how is it relevant to what you want to do?
%\section{Guidance}
%\begin{itemize}
%    \item
%      Don't give a laundry list of references.
%    \item
%      Tie everything you say to your problem.
%    \item
%      Present an argument.
%    \item Think critically; weigh up the contribution of the background and put it in context.
%    \item
%      \textbf{Don't write a tutorial}; provide background and cite
%      references for further information.
%\end{itemize}

%==================================================================================================================================
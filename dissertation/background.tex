In this chapter, we research a suitable implementation language and existing attempts to replace the BSD Sockets API .
First, we discuss the BSD Sockets API and it's limitations.
Then, we discuss three previous attempts to replace BSD Sockets: Post Sockets, NEAT and TAPS.
Next, we discuss a modern, systems program language which will be used to implement the BSD Sockets replacement.

\section{BSD Sockets}\label{sec:bsd-sockets}
The Berkeley Socket API (BSD Sockets) allows an application to communicate over the Internet since 1983.
A key reason for its longevity is how a network socket is treated in the same way as a local file, once the socket has
been created.
This allows a programmer with understanding of local file I/O to work with network sockets.
While this model of network I/O has some similarities to local file I/O, in reality they are distinct and should be
treated as such.
One prime example of this is how network I/O has a range of protocols which can be used at each layer of the networking
stack, each providing different grantees and properties, and the application must specify which protocol stack to use.
Whereas in file I/O, there exists a protocol stack (file systems, storage interface, etc), but the application does not
need to manipulate the stack, as each protocol is a platform-specific way to handle the same operations.

Since BSD Sockets required applications to decide the protocol stack, changing protocols would require the application
to be modified.
This may have contributed to the slow adoption of IPv6, since every application would have to be rewritten to use IPv6
rather than only the network implementation.
BSD Sockets defined the default protocol to handle streams and datagrams are
TCP\footnote{Transmission Control Protocol - IETF RFC793} and UDP\footnote{User Datagram Protocol - IETF RFC768},
respectively.
While this may have aided in the adoption of these protocols and the growth of the Internet, it also hindered the
development of alternative protocols such as DCCP\footnote{Datagram Congestion Control Protocol - IETF RFC4340} and
SCTP\footnote{Stream Control Transmission Protocol - IETF RFC4960}.

BSD Sockets primarily exposes a synchronous API, which does not match the nature of a network and does not reflect
modern network programming techniques.
While there is a non-blocking mode for BSD Sockets, it is error-prone and cumbersome to use effectively.
Furthermore, modern network programming utilises callbacks or futures, which BSD Sockets has no support for.
A custom event loop needs to be created and each socket needs to be polled by the application, rather than the API's
implementation.

BSD Sockets was designed for the Unix operating system, which was implemented in the C programming language, therefore
was implemented in C and exposed a C API.
At the time, C was the correct choice for language as it allowed for easy porting to other platforms, and high
performance.
C has many weaknesses, such as limited concurrency support, widespread undefined behaviour and a weak type system.
These weaknesses can be seen in the design of BSD Sockets: single-threaded synchronous, integer-based API .
Using C can lead to a number of bugs which cannot be checked by the compiler.
For example, a listening socket and a connected socket both have the same type (int), which can lead to misusing a
listening socket in a \texttt{recv()} call, where the connected socket, the listener has produced from an
\texttt{accept()} call, should have been used.
This can lead to a program which blocks indefinitely as a listening socket will never receive data which would unblock
a \texttt{recv()} call.
If the listening socket and the connected socket had different types, the compile could catch such an issue at compile
time.

\section{Post Sockets}\label{sec:post-sockets}
Post Sockets~\citep{kuhlewind_postsocketsabstract_} is an API intended to provide an asynchronous, message-orientated,
transport-protocol independent networking API to be used instead of BSD Sockets (Sockets).
\citet{kuhlewind_postsocketsabstract_} identified Sockets requires the user to tightly couple their program with a
specific transport-protocol.
This lead Post Sockets to be designed, so the user relies on a set of semantics, which could be upheld by an arbitrary
protocol, rather than a specific protocol.

As~\cite{kuhlewind_postsocketsabstract_} explains, modern networked devices have multiple interfaces, which would
provide natural support for multipath communications using protocols such as
MPTCP\footnote{Multipath TCP Extensions - IETF RFC6842}.
The existing sockets API provides limited support for multipath communications, whereas Post Sockets has explicit
support through its \emph{Configuration} object.
Post Socket's \emph{Configuration} objects allow the user to specify a collection of possible protocol stack
configurations.
This provides the user with explicit control over the possible protocol stacks used, and the options each stack has.

Rather than providing a byte-stream orientated API like sockets, Post Sockets exposes a message-orientated API .
Typed messages would be given to or received from a \emph{Carrier} (protocol stack-independent object which sends and
receives messages) as this allows both byte-stream and datagram oriented transport protocols to be supported.

Another point raised by~\cite{kuhlewind_postsocketsabstract_} is layering encryption over the existing sockets API .
Opening a TCP connection with
TLS\footnote{Transport Layer Security - IETF RFC8446} requires two handshakes to happen sequentially: the
transport-layer handshake and the encryption-layer handshake, which increases latency.
To avoid this Post Sockets defines an \emph{Association} object, which contains long-term state between a local endpoint
and a remote endpoint.
This can contain data such as pre-shared keys and cryptographic session resumption parameters.
New \emph{Carrier}s can be created from an existing \emph{Association} object, removing the need to repeat an
encryption-layer handshake between the same local endpoint and remote endpoint.

\section{NEAT}\label{sec:neat}
The NEAT API~\citep{khademi_neatplatformprotocolindependent_2017} is a library to provide a platform and protocol
agnostic network access.
\citet{khademi_neatplatformprotocolindependent_2017} identifies transport-layer protocols are implemented within the
operating system's kernel, and as such are bound by the release cycle of the kernel.
NEAT has been designed as a user-space library, rather then a kernel-space library like Sockets, to allow for new
transport-layer protocols to be included, irrespective of the kernel release cycle.

The duration of kernel release cycles is one of the reasons alternative transport layer protocols have limited uptake.
To add a new transport protocol to sockets, a new kernel release must be developed with this new protocol
NEAT is a user-space library, therefore supports both kernel-space and user-space transport protocols.
User-space transport protocols can be implemented and included in NEAT, independent of the kernel release cycle.
This removes the need to implement new transport protocols over UDP,
removing the often duplicated code to use UDP as a substrate for the protocol.

Existing network applications must specify the protocols they wish to use.
NEAT does not require the application to do this, rather NEAT determines a collection of suitable protocols internally
either from a set of allowed protocols from the user or from all supported protocols.
The method used to select the optimal protocol stack is called Happy Eyeballs~\citep{pauly_happyeyeballsversion_}.
This allows for alternative protocols, including user-space protocols, to be attempted while using traditional protocols
as fallbacks.

\section{TAPS}\label{sec:taps}
Transport Services~\citep{pauly_architecturetransportservices_2020} is proposed standard from the IETF TAPS working
group.
\citet{pauly_architecturetransportservices_2020} identified BSD Sockets did not allow sockets which conceptually
communicated the same data, could not make use of the same API calls.
For example, an encrypted TLS stream could not use the same \texttt{send()} and \texttt{recv()} calls as an unencrypted
TCP stream.
Therefore, TAPS was designed to provide a unified API for all connections which communicate the same data conceptually.

\begin{figure}
    \begin{verbatim}
    Pre-Establishment     :       Established             : Termination
    -----------------     :       -----------             : -----------
                          :                               :
+-- Local Endpoint        :           Message             :
+-- Remote Endpoint       :    Receive() |                :
+-- Transport Properties  :       Send() |                :
+-- Security Parameters   :              |                :
|                         :              |                :
|               InitiateWithSend()       |        Close() :
|   +---------------+   Initiate() +-----+------+ Abort() :
+---+ Preconnection |------------->| Connection |-----------> Closed
    +---------------+ Rendezvous() +------------+         :
   Listen() |             :           |     |             :
            |             :           |     v             :
            v             :           | Connection        :
    +----------+          :           |   Ready           :
    | Listener |----------------------+                   :
    +----------+  Connection Received                     :
                          :                               :
    \end{verbatim}
    \caption{Diagram to show the lifetime of a Connection object
            (Source:~\citet[fig:4]{pauly_architecturetransportservices_2020})}
    \label{fig:tapsConLife}
\end{figure}

TAPS is designed around the lifecycle of a \connection{} object, which is a representation of one or more transport
protocols that send and recieve data between a local and remote machine.
As shown in Figure~\ref{fig:tapsConLife}, a \connection{} object has three stages in its lifecycle: Pre-Establishment,
Established and Termination.

Pre-Establishment is the stage where the \connection{} is created, either directly via the \preconnection{} object's
\texttt{Initiate()}, \texttt{InitiateWithSend()} and \texttt{Rendezvous()} methods, or when a \listener{} receives a
\connection{}.
The \texttt{Initiate()} method is a traditional connection establishment between a local and remote,
\texttt{InitiateWithSend()} method allows data to be sent with the connection establishment
(0-RTT\footnote{Zero Round Trip Time}), and \texttt{Rendezvous()} is use to establish a peer-to-peer connection.
A \preconnection{} represents a potential \connection{}, containing the parameters of a possible future \connection{}.
It contains the definition of the local and/or remote machine via the Local or Remote Endpoints respectively, the
properties the underlying transport protocol must have, and optional security parameters.
A \listener{} object can be created from a \preconnection{} to listen for incoming \connection{}s which match the
properties the creating \preconnection{} has.

Established is the stage when the \connection{} sends and receives data.
\connection{}s communicate with strongly typed messages, rather than the raw bytes the underlying transport protocols
may use.
A \connection{} produces events for when a message been received and when data has been successfully sent, as well as
to signify errors have occurred.
These events are the mechanism TAPS uses to ensure it's API is asynchronous, but an implementation of TAPS does not have
to expose these events directly to the user.
The implementation is free to expose the asynchronous nature of the API is the method which idiomatic for the language,
such as futures or call backs.

Termination is the stage where the \connection{} is closed and cleaned up.
The user will call the \connection{}'s \texttt{close()} or \texttt{abort()} methods to gracefully or forceably end
the \connection{} and cleanup its resources.

No where in the TAPS API is the user able to specify the underlying transport protocol, Rather they specify the
properties the underlying transport protocol must uphold.
\citet[§~5.2]{trammell_abstractapplicationlayer_2020} has defined a set of properties underlying protocols can have,
and the preference a user can have for a property to be upheld.
The implementation of TAPS creates a list of transport protocols it supports, in order of most to least preferred,
excluding any supported protocols which either do not support a “Require” property or support a “Prohibit” property.
A transport protocol is deemed more preferred over another if it contains more “Prefer” properties and/or less “Avoid”
properties.

Since TAPS aims to provide a unified API for connections which communicate the same data, the same data must be able to
be sent/received on different transport protocols.
The two types of transport protocols are byte-stream orientated and datagram orientated.
Datagram orientated transport protocols have natural message boundaries as the boundaries of the datagrams, but
byte-stream orientated transport protocols lack this.
TAPS defines a \framer{} to define message boundaries for protocols which may or may not have natural message
boundaries.
A simple \framer{} could prepend each message with the length of the message, or define an end-of-message sequence for
example.

\section{Rust}\label{sec:rust}
Rust is a modern systems programming language, which aids developers “write faster, more reliable software”, according
to the Rust book (\cite{kalbnik_rustprogramminglanguage_}).
Rust targets systems level programming, similar to C and C++, but provides much greater safety in comparison.

One major issue with writing system software is ownership.
Who owns a resource, such as memory or file handles, and for how long.
In higher level languages, a garbage collector will ensure no resources live longer than they are needed, often through
the use of reference counting.
Garbage collection does elevate the burden of resource management from the developer, but in return, the developer has
much less control over resources, and the resources will live of a non-deterministic length of time with most garbage
collectors.
Languages like C and C++ require the programmer to manage resources, releasing them manually, leading to unreleased
resources if implemented incorrectly.
Rust enforces correct resource management by delegating all resource management to the compiler, through its borrow
checker.
Only a single entity can own a resource in Rust.
All other entities must use references to interact with the resource.
Furthermore, either a single mutable reference or an arbitrary number of immutable references can exist for a resource
at once.
These references cannot live longer than the entity owning the resource, which is enforced by the compiler.
When a reference goes out of scope, it is freed.
When an entity goes out of scope, there must be no remaining references to the entity, and the resources the entity owns
are freed.
Through this method, invalid references cannot exist, and it is impossible to release a resource which has already been
released.
A side effect of only allowing a single mutable reference, or an arbitrary number of immutable references, the compiler
can guarantee a compiled program is data race free.
This is because, only a single entity can modify a resource, while knowing no other entities can read the resource or
modify the resource.

Being a modern language, Rust provides some higher level concepts and construct to systems programming.
While Rust is a statically typed language, it provides local type inference, allowing for cleaning code.
Its type-system is inspired by other languages such as Haskell, as such a programmer can express a large amount of
information through the types they use alone.
Furthermore, Rust has support for higher-order functions, because functions are a first-class type.
Functions, such as map and fold, are provided, not directly for collections of items, rather iterators.
Iterators in Rust provide a functional method for working with collections of items, while providing performance similar
to a comparable loop construct.

Rust also emphasises its ability to provide “zero-cost abstractions”.
This means using high level constructs, such as iterators, do not inure any performance cost over using lower-level
constructs, allowing for code to be written faster and cleaner, without any performance impact.
Another example is Rust's Option type, which is either Some wrapping an object, or None.
This can be the equivalent to a \texttt{NULL}-pointer check in C, due to Rust's null-pointer optimisation, but is much
safer as the type-system forces the programmer to check for the presence of an object.
A Future in Rust is a type which is polled until it is complete.
A key feature of Rust's Futures is they only carry out work when polled, so there is no extra cost to use a Future.
Rather than polling all the Futures manually, a runtime is used to manage the Futures.
The runtime is responsible for distributing the Futures among its worker poll if applicable, and ensuring each Future is
polled.

Recently, Rust introduced support for \texttt{async/await} syntax~\citep{withoutboats_asyncawaitnotation_}.
A function can be marked as \texttt{async}, which changes the return type to be a \texttt{Future} wrapping the original
return type.
When an \texttt{async} function reaches an \texttt{await}, the poll being used to progress the Future returns Pending,
informing the runtime this \texttt{Future} is incomplete and another \texttt{Future} should be polled.
The \texttt{Future} will continue to return \texttt{Pending} until the \texttt{Future} being awaited completes,
at which point the function will continue until the next \texttt{await} or the function returns.
The new syntax allows for asynchronous code to be written quickly and cleanly with no runtime cost, compared too
manually handling \texttt{Future}s.

\section{Summary}\label{sec:summary}
BSD Sockets has become less applicable as a networking API since its creation, with the main issues being: being a
predominantly synchronous API, being an integer-based API, forcing the user to tightly couple their program to
transport protocols and limiting the uptake of new transport protocols.
A solution is to create a new API which builds upon the key ideas of the existing attempts to replace BSD Sockets,
resulting in a high-level, strongly typed, asynchronous API .
This new API will make use of a modern systems programming language (Rust) to add more compile-time safety, while having
minimal impact on performance.
Furthermore, the new API will enforce weak-coupling between the program and the transport protocols by not allowing the
user to specify specific transport protocols, rather specify preferences on a set of properties which a transport
protocol must uphold.

%What did other people do, and how is it relevant to what you want to do?
%\section{Guidance}
%\begin{itemize}
%    \item
%      Don't give a laundry list of references.
%    \item
%      Tie everything you say to your problem.
%    \item
%      Present an argument.
%    \item Think critically; weigh up the contribution of the background and put it in context.
%    \item
%      \textbf{Don't write a tutorial}; provide background and cite
%      references for further information.
%\end{itemize}

%==================================================================================================================================